% Inbuilt themes in beamer
\documentclass{beamer}

\usepackage{graphicx}
\usepackage{tikz}

\usepackage{hyperref}
\usepackage{color}
\hypersetup{
    colorlinks=true,
    linkcolor=blue,
    filecolor=magenta,      
    urlcolor=cyan,
}

% Theme choice:
\usetheme{metropolis}

\setbeamercolor{block title alerted}{fg=white, bg=orange}
\setbeamercolor{block body alerted}{ bg=orange!25}
% Change standard block colors
\setbeamercolor{block title}{bg=cyan, fg=white}
\setbeamercolor{block body}{bg=cyan!10}

% Title page details: 
\title{Cyberpunk RED} 
\author{Einführung}
\date{\today}
%\logo{
%  \begin{tikzpicture}[overlay,remember picture]
%    \node[left=1cm,below=0.001cm] at (current page.33){
%        \includegraphics[width=1.5cm]{assets/cpred_logo.png}
%    };
%  \end{tikzpicture}
%  %\includegraphics[scale=0.18]{assets/cpred_logo.png}
%}

\addtobeamertemplate{sidebar right}{}{%
    \begin{tikzpicture}[remember picture, overlay]
        \node[
            anchor=north east, % Anchor the node at its top-right
            xshift=-0.3cm,     % Move left from edge
            yshift=0.3cm      % Move down from edge
        ] at (current page.north east) {%
            \includegraphics[height=1.5cm]{assets/cpred_logo.png}
        };
    \end{tikzpicture}%
}
\addtobeamertemplate{frametitle}{}{%
    \begin{tikzpicture}[remember picture, overlay]
        \node[
            anchor=north east, % Anchor the node at its top-right
            xshift=-0.2cm,     % Move left from edge
            yshift=0.4cm      % Move down from edge
        ] at (current page.north east) {%
            \includegraphics[height=1.5cm]{assets/cpred_logo.png}
        };
    \end{tikzpicture}%
}

\begin{document}

% Title page frame
\begin{frame}
    \titlepage 
\end{frame}

% Remove logo from the next slides
%\logo{}

% Outline frame
\begin{frame}{Outline}
    \tableofcontents
\end{frame}

\section{Ablauf}
\begin{frame}
  \begin{enumerate}
    \item PnP Kurzerklärung 
    \item Session 0 (Checkliste)
    \item Einführung in die Welt von Cyberpunk RED
    \item Zusammenfassung des Spielsystems (und wichtiger Regeln)
    \item Charaktererstellung (+FoundryVTT)
    \item \textbf{(Intro) Abenteuer}
    \item \textbf{Feedbackrunde}
  \end{enumerate}  
\end{frame}

\section{Pen and Paper Basics}
\begin{frame}{Was ist das?}
  \begin{itemize}
    \item Rollenspiel (oder TTRPG)
    \item Mischung aus (Brett-)spiel, Erzählung, Improvisation, lediglich eingeschränkt durch ein Regelsystem
    \item Geschichte entsteht während des Spielens
    \item Ursprünglich aus \textit{Wargames} in den 1950ern entstanden, mit Fantasy Einflüssen
    \item Veröffentlichung von \textit{Dungeons \& Dragons} 1974
    \item Erstes professionelles, \textbf{deutsches} Abenteuer 1984 mit DSA
  \end{itemize}
\end{frame}

\begin{frame}{Genres \& Spielsysteme}
  \begin{itemize}
    \item Fantasy als größtes Genre, z.B. D\&D, DSA, Aborea
    \item ...aber auch Sci-Fi (Shadowrun), Horror/Mystery (Call of Cthulhu)
    \item Regelwerke legen Vorgaben für den Spielablauf fest (auch generische \textit{Universalwerke} möglich)
    \item Verschiedene Prinzipien
  \end{itemize}
  \begin{block}{Teilnehmer}
    \textbf{Gruppenabenteuer} bestehen im Normalfall aus einem Spielleiter und mehreren Spielern, welche Rollen unterschiedlicher Charaktere übernehmen, während der Spielleiter die Beschreibung der Welt, sowie Nicht-Spieler-Charaktere übernimmt.
  \end{block}
\end{frame}

\begin{frame}
  \begin{figure}
    \includegraphics[scale=0.2]{assets/player_sceptic.jpeg}    
  \end{figure}  
\end{frame}


\begin{frame}{Ablauf}
  \begin{itemize}
    \item Jeder Spieler erhält (normalerweise) einen Stift, Papier (mit Charakterinformationen), und verschiedene Würfel
    \item Der Spielleiter bereitet ein Szenario vor, in welchem sich die Spieler befinden und beschreibt dieses den Spielern
    \item Die Spieler treffen die Entscheidungen, der Spielleiter reagiert (im Rahmen des Regelwerks)
    \item Bestimmte, durch den Spielleiter festgelegte, Aktionen, müssen durch einen Würfelwurf bestätigt werden
  \end{itemize}
\end{frame}

\begin{frame}{Ablauf}
  \begin{block}{Game Loop}
    \begin{enumerate}
      \item \textbf{Situation}: Leitung beschreibt die Szene
      \item \textbf{Aktion}: Spieler tun Dinge bzw. versuchen es
      \item \textbf{Probe}: Würfel tut Dinge
      \item \textbf{Konsequenz}: Leitung beschreibt das Ergebnis
    \end{enumerate}
  \end{block}
  \begin{block}{Kämpfe}
    Festlegung einer Aktionsreihenfolge der Charaktere.
    \begin{enumerate}
      \item \textbf{Start einer Runde}: in Zeiteinheiten unterteilt
      \item \textbf{Ausführen von Aktionen}: z.B. Bewegen, Angreifen, Dinge aufheben
      \item \textbf{Beenden einer Runde}: der nächste Charakter führt seine Aktion aus, danach ist eine Runde beendet
    \end{enumerate}
  \end{block}
\end{frame}

\section{Cyberpunk RED: Session 0}
\begin{frame}{Orga}
  \begin{itemize}
    \item Nächste Termine
    \item Spielstil
    \item Was wird gespielt
    \item Erwartungen
  \end{itemize}
  $\rightarrow$ \url{https://rollenspielblog.net/rollenspiel/session-0/\#Session_0_Checkliste}
\end{frame}

\section{Cyberpunk RED: Infos}
\begin{frame}{Cyberpunk RED}
  \begin{alertblock}{Was ist Cyberpunk?}
    \begin{itemize}
        \item \textbf{Ursprung:} Mix aus 60er/70er Sci-Fi (Gibson, Dick) und japanischer Popkultur; inspiriert von Film Noir.
        \item \textbf{Szenario:} ,,High Tech, Low Life`` -- Fortgeschrittene Technologie in einer post-industriellen, dystopischen Welt.
        \item \textbf{Machtstruktur:} Dominanz von (großen) Konzernen
        \item \textbf{Mensch \& Maschine:} Körpermodifikationen
    \end{itemize}
  \end{alertblock}
  \begin{block}{TTRPG}
    \begin{itemize}
      \item 1988: Cyberpunk 2013 TTRPG, Entwickelt von Mike Pondsmith
      \item ...1990: Cyberpunk 2020, 2005: Cyberpunk 3.0, 2020: Cyberpunk RED
    \end{itemize}
  \end{block}
\end{frame}

\begin{frame}{Night City im Jahr 2045}
    \begin{itemize}
        \item \textbf{Der Ort:} Night City ist ein ,,Freistaat`` an der Grenze zwischen Nord- und Südkalifornien -- unabhängig von den zerfallenen NUSA (New USA).
        \item \textbf{Die Wirtschaft:} 
            \begin{itemize}
                \item Mangelwirtschaft durch unterbrochene Lieferketten.
                \item \textbf{Night Markets:} Illegale Pop-up-Märkte in Containern/Ruinen. Nur über ,,Fixer`` zu finden.
            \end{itemize}
        \item \textbf{Lebensstandard:}
            \begin{itemize}
                \item Echtes Essen ist Luxus. Standard ist ,,Kibble`` (Trockenfutter) oder ,,Scop`` (Algen/Pilzmasse).
                \item \textbf{Combat Zones:} Gesetzlose Zonen, die von der Polizei (NCPD) aufgegeben wurden.
            \end{itemize}
    \end{itemize}
\end{frame}

% Folie 2: Technologie & Netz
\begin{frame}{Das Netz \& Technologie}
    \begin{itemize}
        \item \textbf{Das Netz (Nach dem DataKrash):}
            \begin{itemize}
                \item Kein weltweites Hacking mehr möglich.
                \item Netzwerke sind isoliert (Air-gapped). \textbf{Netrunner} müssen physisch vor Ort sein (Kabel oder 6m Reichweite).
                \item \textbf{CitiNet:} Ein lokales, textbasiertes Netzwerk für die Stadt.
            \end{itemize}
        \item \textbf{Cyberware \& Menschlichkeit:}
            \begin{itemize}
                \item Körperteile können durch Maschinen ersetzt werden.
                \item \textbf{Preis:} Verlust von Empathie (Humanity).
                \item Sinkt die Menschlichkeit auf 0, droht die \textbf{Cyberpsychose}: Der Verlust der Kontrolle und Wahnsinn (Zielscheibe für MaxTac/PsychoSquad).
            \end{itemize}
    \end{itemize}
\end{frame}

% Folie 3: Machtgruppen
\begin{frame}{Konzerne \& Gangs}
    \begin{itemize}
        \item \textbf{Megakonzerne:}
            \begin{itemize}
                \item \textbf{Militech:} Waffenproduktion \& Privatarmee der NUSA; wollen Night City annektieren.
                \item \textbf{BioTechnica:} Produzenten von CHOOH2 (Treibstoff) und Nahrung.
                \item \textbf{Arasaka:} Japanischer Rüstungsgigant. Offiziell aus der Stadt verbannt, operiert aber im Schatten.
            \end{itemize}
        \item \textbf{Gangs (Beispiele):}
            \begin{itemize}
                \item \textbf{Maelstrom:} Technik-obsessiv, instabil, entfernen fast alles Fleisch.
                \item \textbf{Tyger Claws:} Organisiertes Verbrechen, asiatisch geprägt, ,,ehrenhaft``.
                \item \textbf{Bozos:} Letale Bio-Sculpting Clowns (Terroristen).
                \item \textbf{Inquisitoren:} Kult, der Cyberware als Sünde ansieht.
            \end{itemize}
    \end{itemize}
\end{frame}

% Folie 4: Historischer Kontext (Kurzfassung)
\begin{frame}{Timeline (kurz)}
    \begin{itemize}
        \item \textbf{Der Kollaps (1990-2020):} Zusammenbruch der USA, Aufstieg der Megakonzerne.
        \item \textbf{Der Fall von Night City (2023):}
            \begin{itemize}
                \item Ein Team (inkl. Johnny Silverhand) infiltriert den Arasaka Tower.
                \item Eine nukleare Explosion zerstört das Stadtzentrum.
                \item Der Himmel färbt sich jahrelang rot $\rightarrow$ \textit{Time of the Red}.
            \end{itemize}
        \item \textbf{Die Gegenwart (2045):}
            \begin{itemize}
                \item Phase des Wiederaufbaus. Der Himmel ist nicht mehr blutrot, aber smog-grau.
                \item Arasaka ist noch weg, Militech lauert an der Grenze.
                \item Night City kämpft um seine Unabhängigkeit.
            \end{itemize}
    \end{itemize}
\end{frame}

\begin{frame}{Prinzipien von Cyberpunk RED}
    \begin{alertblock}{The Three Laws of Cyberpunk}
        \begin{enumerate}
          \item \Large \textbf{Style over Substance}
          \item \Large \textbf{Attitude is Everything}
          \item \Large \textbf{Live on the Edge}
        \end{enumerate}
    \end{alertblock}
\end{frame}

\begin{frame}{Beachten}
  \begin{alertblock}{Zusammengefasst:}
    \begin{itemize}
      \item Charaktere können \textbf{sehr schnell} sterben
      \item $\rightarrow$ die Story steht hier klar im Vordergrund
      \item Grundsätzlich spielt jeder für sich selbst und das Überleben/Lifestyle seines Charakters
      \item es gibt (abgesehen von mehrstufigen Abenteuern) keine grundlegende größere Story...
      \item ...es wird von Auftrag zu Auftrag in variierender Größe erzählt
    \end{itemize}
  \end{alertblock}
\end{frame}

\section{Cyberpunk RED: Spielsystem(e)}
\begin{frame}{Generelle Infos}
  \begin{itemize}
    \item Würfel: d10 + (5-6) d6, mehr nicht
    \item Skill basiertes System, keine Levelaufstiege (eigentlich immer eigener Check vs. Gegner)
    \item Standard Check: \textbf{STAT + Skill + 1d10} (vs. gegner check oder gegen DV (Difficulty value)), muss höher sein!
    \item Kritische Würfe: \textbf{10} oder \textbf{1} jeweils nochmal 1d10 würfeln und auf Wert addieren bzw. subtrahieren
    \item Jedem Skill ist jeweils ein STAT zugeordnet, GM bestimmt welcher Skill und DV genutzt und geschafft werden muss (inkl. mögl. Modifier)
  \end{itemize}
\end{frame}

\begin{frame}{STATs}
  \begin{itemize}
    \item \textbf{Mental}: INT, WILL, COOL, EMP
    \item \textbf{Combat}: TECH, REF
    \item \textbf{Fortune}: LUCK (begrenzt pro Session)
    \item \textbf{Physical} BODY, DEX, MOVE
    \item Sonst: Humanity (10*EMP, wechselwirkend), Hit Points (10 + (5[BODY und WILL average])
  \end{itemize}
\end{frame}

\begin{frame}
  \begin{figure}
    \includegraphics[scale=0.1]{assets/rpg_progression.png}    
  \end{figure}  
\end{frame}


\begin{frame}{Skills}
  \begin{itemize}
    \item Skills (und Abilities) können mit \textbf{Improvement Points} gelernt/verbessert werden (Abilities jeweils nur 1 Stufe pro)
    \item DV (Difficulty Value) hängt von der Situation ab
    \item \textbf{Sehr große Auswahl an Skills}
    \item Gelernte Skills starten immer mit Stufe 2, Liste an Basic Skills (z.B. Sprache) vorgegeben
    \item Unterteilt in verschiedene Kategorien (z.B. Awareness, Body, Social...)
    \item Manche Skills erfordern mehr IP bzw. Punkte \textbf{(x2)}
    \item Teilweise beeinflusst durch Abilities der Rollen
  \end{itemize}
\end{frame}

\begin{frame}{(Role) Abilities und Lifepath}
  \begin{block}{Abilities}
    ...als einzige Rollen-basierte Unterscheidung, z.B. Netrunner mit Interface kann NET rennen, Execs bekommen NPCs als Team und ein kostenloses Apartment usw. (mehr dazu)
  \end{block}
  \begin{block}{Lifepath}
    \begin{itemize}
      \item Lifepath als Hintergrund "System" für Story Hooks/Plot Points
      \item Unterteilung in generelle und Rollen-basierte Lifepath Entscheidungen
      \item Beinhaltet sowas wie Freunde, Feinde, Motivation, Reputation
      \item Reputationssystem, basierend auf Aktionen (Facedowns: \textbf{COOL + 1d10 + Reputation})
    \end{itemize}
  \end{block}
\end{frame}

\begin{frame}{Generelle Infos}
  \begin{itemize}
    \item Eher auf Story/Action ausgelegt
    \item Starten mit Würfeln von \textbf{Initiative: REF + 1d10}, absteigend sortiert
    \item Eine Runde: 3 Sek., jeweils 1 Zug aller Teilnehmenden
    \item Ein Zug: 1 Bewegung (\textbf{MOVE} = 1square) + 1 Aktion (z.B. Nachladen, Angreifen etc.), kann aber kombiniert werden (z.B. l, s, l, s, l)
    \item Besteht aus: Nah-/Fernkampf, Brawling, Grappling, Martial Arts, (Netrunning)
    \item Schaden wird genommen nach Abzug von Rüstung, sowie kritische Verletzungen
    \item Deckungssystem (nicht anvisierbar falls \textbf{komplett} hinter einer Deckung z.B. Wand)
  \end{itemize}  
\end{frame}

\begin{frame}{Fernkampf (Ranged)}
  \begin{block}{Basic Check}
    \textbf{REF + Weapon Skill + 1d10} vs \textbf{DV of Defender by Range (Target/Weapons)} oder DEX + Evasion + 1d10 (falls Defender REF \>= 8)    
  \end{block}
  \begin{block}{Called Shots}
      sog. \textbf{Aimed Shots} vorher angekündigt auf Körperteil mit \textbf{-8} Modifier, dafür zus. Effekte
  \end{block}
\end{frame}

\begin{frame}{Fernkampf Part 2}
  \begin{block}{Modi}
    Versch. Feuermodi (z.B. durch Autofire Skill), min. 10 Kugeln im Magazin, 2d6 x DV to hit, (siehe AF chart), oder Suppressive Fire (auch min. 10 Ammo innerhalb 20sq, WILL + Concentration + 1d10 check vs your REF + Autofire + 1d10)
  \end{block}
  \begin{block}{Sonstiges}
    Bogen (Nachladen keine eigene Aktion), Shotguns (alle vor einem DV13, 3sq), Explosives (in 5x5sq.)
  \end{block}
\end{frame}

\begin{frame}
  \begin{figure}
    \includegraphics[scale=0.25]{assets/shotguns.jpeg}    
  \end{figure}  
\end{frame}

\begin{frame}{Nahkampf (Melee)}
  \begin{block}{Generell}
    Innerhalb 2sq, \textbf{DEX + Relevant Melee Att. Skill + 1d10} vs \textbf{Def. DEX + Evasion + 1d10} (melee ignore half def. armour, round up)    
  \end{block}
  \begin{block}{Brawling, Bar Knuckle, Grappling}
    \begin{itemize}
      \item Skaliert mit BODY: Grapple, Snatch, Chocke, Throw möglich (mit \textbf{DEX + Brawl + 1d10 vs Def. DEX + Brawl + 1d10}), -2 für alle Aktionen, Gegner kein MOVE
      \item Schilde möglich (Einhändig, Gegner)      
    \end{itemize}
  \end{block}
  \begin{block}{Martial Arts}
    \textbf{DEX + MA Skill + 1d10 vs. Def. DEX + Evasion + 1d10 or DV}, 2ROF, je nach Martial Arts Skill Special Moves
  \end{block}
\end{frame}

\begin{frame}
  \begin{figure}
    \includegraphics[scale=0.4]{assets/death_saves.jpg}    
  \end{figure}  
\end{frame}

\begin{frame}{Schaden}
  \begin{block}{Schaden nehmen}
    \begin{enumerate}
      \item Angreifer würfelt Schaden      
      \item Armor SP (norm. am Körper) abziehen und Rest von HP abziehen (mit Ausnahmen)
      \item Falls Schaden genommen, von Armor 1 SP abziehen
    \end{enumerate}
  \end{block}
  \begin{block}{Wundstatus}
    Leicht Verletzt (kleiner Full HP), Stark Verletzt (kleiner 1/2 HP), Tödlich Verletzt (kleiner 1 HP). Stabilisierung über DV erreichbar
  \end{block}
  \begin{block}{Death Save}
    \textbf{1d10} muss unter BODY liegen, jede Runde um 1 erhöht. DV15 für Stabilisierung.
    \textbf{Ein nicht geschaffter DS $\rightarrow$ Tod.}
  \end{block}
\end{frame}

\begin{frame}{Armor, Kritische Verletzungen}
  \begin{block}{Armor}
    SP (Stopping Power), unterteilt in Head und Body Armor
  \end{block}
  \begin{block}{Kritische Verletzungen}
    Auftreten, falls von Angriff Schaden 2x6 gewürfelt wird. Dann Critical Injury Table nutzen und 2d6.
  \end{block}  
  \begin{block}{Sonstiges}
    Verbrennen, Ertrinken (1min = 1BODY), Fallschaden, Drogen/Gifte, Radioaktivität
  \end{block}
\end{frame}

\begin{frame}{Heilung usw}
  \begin{block}{Trauma Team}
    Bewaffnetes Ärzte Team kommt vorbei, falls Abo abgeschlossen wurde (Silber 500eb/Exec 1000eb pro Monat) oder für Teammitglied. 1d6 Runden bis die da sind.
  \end{block}
  \begin{block}{Stabilisierung und Heilung}
    Stabilisierung beginnt den Heilungsprozess (HP = BODY pro Tag wenn man sich ausruht) durch DV10, 13, 15 von Ziel durch \textbf{TECH + First Aid/Paramedic + 1d10}, bei 1HP erstmal bewusstlos
  \end{block}
  \begin{block}{Heilung durch Skills}
    Möglich mit \textbf{Cybertech, First Aid, Paramedic, Surgery}, z.B. Quick Fix 1min (Injury weg, Effekt bleibt) oder Treatment 4hr (permanent, nur extern möglich). Sonst auch Krankenhaus möglich.
  \end{block}
\end{frame}

\begin{frame}{Cyberware, Cyberpsychosis}
  \begin{block}{Installation}
    Nicht alle Cyberware kann überall installiert werden (siehe Tabelle). Bodysculpting möglich zur Veränderung des Aussehens.
  \end{block}
  \begin{block}{Cyberpsychosis}    
    Installationen von Cyberware (oder auch Traumata) reduzieren Humanity (und damit EMP), was bei 0 zu einer Cyberpsychosis führen kann und der Charakter vom GM übernommen und von der Polizei gejagt wird. Therapie kann diese und Folge von Drogen wieder abschwächen.
  \end{block}
\end{frame}

\begin{frame}
  \begin{figure}
    \includegraphics[scale=0.5]{assets/netrunning.jpeg}    
  \end{figure}  
\end{frame}

\begin{frame}{Netrunning}
  \begin{alertblock}{Hinweise}
    Zusätzliche NET Ebene in Kämpfen und Situationen. Es muss sich zunächst physisch mit dem Interface des jeweiligen lokalen Netzwerks verbunden werden. Nur möglich für Netrunner (Wizard Äquivalent)!
  \end{alertblock}
  \begin{itemize}    
    \item NET Ebenen sind quasi wie ein Gebäude aufgebaut mit verschiedenen Ebenen
    \item Verschiedene Abwehrmechanismen (z.B. Black ICE) oder Angreifer sind dort vorhanden
    \item Programme des Cyberdecks nutzen für Aktionen im NET
    \item Es können z.B. Kameras, Fallen aktiviert werden
    \item Je "Meat Space" Runde, Anzahl an NET Aktionen (basiert auf Interface Rang)
    \item \textbf{Interface + (BI/Prog) ATK + 1d10} vs Gegner
  \end{itemize}
\end{frame}

\begin{frame}{Sonstiges}
  \begin{alertblock}{Hinweis}
    Erstmal nicht ganz so wichtig, nur der Vollständigkeit halber kurz aufgeführt!
  \end{alertblock}
  \begin{itemize}
    \item Fahrzeugkämpfe (versch. Fahrzeuge, Maneuver, Rammen und Ausweichen)
    \item Hustle Charts (was passiert zwischen Sessions)
    \item Lifestyle (man braucht halt monatlich Geld, Schlafen ist wichtig)
  \end{itemize}
\end{frame}

\section{Cyberpunk RED: Charaktererstellung}
\begin{frame}{Charaktererstellung}
  Verschiedene Möglichkeiten (Template, halbes Template, "Calculated"), generell:
  \begin{enumerate}
    \item Rolle wählen (Abstimmen in Gruppe/Meet the Teams)
    \item Lifepath wählen/würfeln
    \item STATS, derived STATs würfeln/wählen
    \item Skills würfeln/wählen
    \item Gear, Outfit, Lifestyle, Cyberware würfeln/kaufen
  \end{enumerate}
  $\rightarrow$ \href{https://cyberpunkred.com/}{Charaktererstellung online (dann Export)}
\end{frame}

\section{\textbf{Intro Abenteuer}}
\begin{frame}{FoundryVTT}
  \begin{block}{Virtual Table Top}
    Bessere Übersicht, digital, Spielsystem zum Klicken (einfacher am Anfang).        
    \\
    $\rightarrow$ \href{https://foundryvtt.com/}{Hier Link einfügen}
  \end{block}  
\end{frame}

\section{\textbf{Feedbackrunde}}
\begin{frame}{Feedbackrunde}
  \begin{itemize}
    \item "Stars": Was war besonders gut? (min. 1)
    \item Wünsche: z.B. für nächstes Mal
    \item (Allgemeines (Spiel an sich, GM, "Ich bin raus"))
  \end{itemize}
\end{frame}

\end{document}
